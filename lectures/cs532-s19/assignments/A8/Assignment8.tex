%%%%%%%%%%%%%%%%%%%%%%%%%%%%%%%%%%%%%%%%%
% Programming/Coding Assignment
% LaTeX Template
%
% This template has been downloaded from:
% http://www.latextemplates.com
%
% Original author:
% Ted Pavlic (http://www.tedpavlic.com)
%
% Note:
% The \lipsum[#] commands throughout this template generate dummy text
% to fill the template out. These commands should all be removed when 
% writing assignment content.
%
% This template uses a Perl script as an example snippet of code, most other
% languages are also usable. Configure them in the "CODE INCLUSION 
% CONFIGURATION" section.
%
%%%%%%%%%%%%%%%%%%%%%%%%%%%%%%%%%%%%%%%%%

%----------------------------------------------------------------------------------------
%	PACKAGES AND OTHER DOCUMENT CONFIGURATIONS
%----------------------------------------------------------------------------------------

\documentclass{article}

\usepackage{fancyhdr} % Required for custom headers
\usepackage{lastpage} % Required to determine the last page for the footer
\usepackage{extramarks} % Required for headers and footers
\usepackage[usenames,dvipsnames]{color} % Required for custom colors
\usepackage{graphicx} % Required to insert images
\usepackage{listings} % Required for insertion of code
\usepackage{courier} % Required for the courier font
\usepackage{lipsum} % Used for inserting dummy 'Lorem ipsum' text into the template
\usepackage{setspace}
\usepackage{color}
\usepackage{comment}
\usepackage{caption}

\usepackage{hyperref}
\usepackage{natbib}
\usepackage{underscore}
\usepackage{subfigure}
\usepackage{fixltx2e}

\hypersetup{
    colorlinks=true,
    linkcolor=blue,
    filecolor=magenta,      
    urlcolor=cyan,
    breaklinks=true
}

%\usepackage[]{algorithm2e}
\usepackage{pdfpages}




%For python inclusion (http://widerin.org/blog/syntax-highlighting-for-python-scripts-in-latex-documents)
\definecolor{Code}{rgb}{0,0,0}
\definecolor{Decorators}{rgb}{0.5,0.5,0.5}
\definecolor{Numbers}{rgb}{0.5,0,0}
\definecolor{MatchingBrackets}{rgb}{0.25,0.5,0.5}
\definecolor{Keywords}{rgb}{0,0,1}
\definecolor{self}{rgb}{0,0,0}
\definecolor{Strings}{rgb}{0,0.63,0}
\definecolor{Comments}{rgb}{0,0.63,1}
\definecolor{Backquotes}{rgb}{0,0,0}
\definecolor{Classname}{rgb}{0,0,0}
\definecolor{FunctionName}{rgb}{0,0,0}
\definecolor{Operators}{rgb}{0,0,0}
\definecolor{Background}{rgb}{0.98,0.98,0.98}

% Margins
\topmargin=-0.45in
\evensidemargin=0in
\oddsidemargin=0in
\textwidth=6.5in
\textheight=9.0in
\headsep=0.25in

\linespread{1.1} % Line spacing

% Set up the header and footer
\pagestyle{fancy}
\lhead{\hmwkAuthorName} % Top left header
%\chead{\hmwkClass\ (\hmwkClassInstructor\ \hmwkClassTime): \hmwkTitle} % Top center head
\chead{\hmwkClass\ (\hmwkClassInstructor): \hmwkTitle} % Top center head
\rhead{\firstxmark} % Top right header
\lfoot{\lastxmark} % Bottom left footer
\cfoot{} % Bottom center footer
\rfoot{Page\ \thepage\ of\ \protect\pageref{LastPage}} % Bottom right footer
\renewcommand\headrulewidth{0.4pt} % Size of the header rule
\renewcommand\footrulewidth{0.4pt} % Size of the footer rule

\setlength\parindent{0pt} % Removes all indentation from paragraphs

%----------------------------------------------------------------------------------------
%	CODE INCLUSION CONFIGURATION
%----------------------------------------------------------------------------------------

\definecolor{MyDarkGreen}{rgb}{0.0,0.4,0.0} % This is the color used for comments
\lstloadlanguages{Perl} % Load Perl syntax for listings, for a list of other languages supported see: ftp://ftp.tex.ac.uk/tex-archive/macros/latex/contrib/listings/listings.pdf
\lstset{language=Perl, % Use Perl in this example
        frame=single, % Single frame around code
        basicstyle=\small\ttfamily, % Use small true type font
        keywordstyle=[1]\color{Blue}\bf, % Perl functions bold and blue
        keywordstyle=[2]\color{Purple}, % Perl function arguments purple
        keywordstyle=[3]\color{Blue}\underbar, % Custom functions underlined and blue
        identifierstyle=, % Nothing special about identifiers                                         
        commentstyle=\usefont{T1}{pcr}{m}{sl}\color{MyDarkGreen}\small, % Comments small dark green courier font
        stringstyle=\color{Purple}, % Strings are purple
        showstringspaces=false, % Don't put marks in string spaces
        tabsize=5, % 5 spaces per tab
        %
        % Put standard Perl functions not included in the default language here
        morekeywords={rand},
        %
        % Put Perl function parameters here
        morekeywords=[2]{on, off, interp},
        %
        % Put user defined functions here
        morekeywords=[3]{test},
       	%
        morecomment=[l][\color{Blue}]{...}, % Line continuation (...) like blue comment
        numbers=left, % Line numbers on left
        firstnumber=1, % Line numbers start with line 1
        numberstyle=\tiny\color{Blue}, % Line numbers are blue and small
        stepnumber=5 % Line numbers go in steps of 5
}

% Creates a new command to include a perl script, the first parameter is the filename of the script (without .pl), the second parameter is the caption
\newcommand{\perlscript}[2]{
\begin{itemize}
\item[]\lstinputlisting[caption=#2,label=#1]{#1.pl}
\end{itemize}
}


%----------------------------------------------------------------------------------------
%	DOCUMENT STRUCTURE COMMANDS
%	Skip this unless you know what you're doing
%----------------------------------------------------------------------------------------

% Header and footer for when a page split occurs within a problem environment
\newcommand{\enterProblemHeader}[1]{
\nobreak\extramarks{#1}{#1 continued on next page\ldots}\nobreak
\nobreak\extramarks{#1 (continued)}{#1 continued on next page\ldots}\nobreak
}

% Header and footer for when a page split occurs between problem environments
\newcommand{\exitProblemHeader}[1]{
\nobreak\extramarks{#1 (continued)}{#1 continued on next page\ldots}\nobreak
\nobreak\extramarks{#1}{}\nobreak
}

\setcounter{secnumdepth}{0} % Removes default section numbers
\newcounter{homeworkProblemCounter} % Creates a counter to keep track of the number of problems

\newcommand{\homeworkProblemName}{}
\newenvironment{homeworkProblem}[1][Problem \arabic{homeworkProblemCounter}]{ % Makes a new environment called homeworkProblem which takes 1 argument (custom name) but the default is "Problem #"
\stepcounter{homeworkProblemCounter} % Increase counter for number of problems
\renewcommand{\homeworkProblemName}{#1} % Assign \homeworkProblemName the name of the problem
\section{\homeworkProblemName} % Make a section in the document with the custom problem count
\enterProblemHeader{\homeworkProblemName} % Header and footer within the environment
}{
\exitProblemHeader{\homeworkProblemName} % Header and footer after the environment
}

\newcommand{\problemAnswer}[1]{ % Defines the problem answer command with the content as the only argument
\noindent\framebox[\columnwidth][c]{\begin{minipage}{0.98\columnwidth}#1\end{minipage}} % Makes the box around the problem answer and puts the content inside
}

\newcommand{\homeworkSectionName}{}
\newenvironment{homeworkSection}[1]{ % New environment for sections within homework problems, takes 1 argument - the name of the section
\renewcommand{\homeworkSectionName}{#1} % Assign \homeworkSectionName to the name of the section from the environment argument
\subsection{\homeworkSectionName} % Make a subsection with the custom name of the subsection
\enterProblemHeader{\homeworkProblemName\ [\homeworkSectionName]} % Header and footer within the environment
}{
\enterProblemHeader{\homeworkProblemName} % Header and footer after the environment
}

%----------------------------------------------------------------------------------------
%	NAME AND CLASS SECTION
%----------------------------------------------------------------------------------------
%#MOD
\newcommand{\hmwkTitle}{Assignment\ \#8 } % Assignment title
%\newcommand{\hmwkDueDate}{Monday,\ January\ 1,\ 2012} % Due date
\newcommand{\hmwkClass}{Web Science} % Course/class
%\newcommand{\hmwkClassTime}{10:30am} % Class/lecture time
\newcommand{\hmwkClassInstructor}{Alexander Nwala} % Teacher/lecturer
\newcommand{\hmwkAuthorName}{Mohd. Nauman Siddique} % Your name

%----------------------------------------------------------------------------------------
%	TITLE PAGE
%----------------------------------------------------------------------------------------

\title{
\vspace{2in}
\textmd{\textbf{\hmwkClass:\ \hmwkTitle}}\\
%\normalsize\vspace{0.1in}\small{Due\ on\ \hmwkDueDate}\\
%\vspace{0.1in}\large{\textit{\hmwkClassInstructor\ \hmwkClassTime}}
\vspace{0.1in}\large{\textit{\hmwkClassInstructor}}
\vspace{3in}
}

\author{\textbf{\hmwkAuthorName}}
%#MOD
\date{Sunday, April 14, 2019} % Insert date here if you want it to appear below your name

%----------------------------------------------------------------------------------------

\begin{document}

\maketitle



%----------------------------------------------------------------------------------------
%	TABLE OF CONTENTS
%----------------------------------------------------------------------------------------

%\setcounter{tocdepth}{1} % Uncomment this line if you don't want subsections listed in the ToC

\newpage
\tableofcontents
\newpage

%----------------------------------------------------------------------------------------
%	PROBLEM 1
%----------------------------------------------------------------------------------------

% To have just one problem per page, simply put a \clearpage after each problem

\begin{homeworkProblem}


 Create two datasets; the first called Testing, the second called Training. 
	
The Training dataset should:
\begin{enumerate}
\item consist of 10 text documents for email messages you consider spam (from your spam folder)
\item consist of 10 text documents for email messages you consider not spam (from your inbox)
\end{enumerate}

The Testing dataset should:
\begin{enumerate}
\item consist of 10 text documents for email messages you consider spam (from your spam folder)
\item consist of 10 text documents for email messages you consider not spam (from your inbox)
\end{enumerate}


\begin{enumerate}
\item Upload your datasets on github
\item Please do not include emails that contain sensitive information
\end{enumerate}

%\problemAnswer
%{
    \begin{verbatim}\end{verbatim}
    \textbf{SOLUTION}\\

I created  a dataset for the email classififcation problem with training and testing data each folder having 10 spam and non-spam emails. The dataset has been uploaded to the Github. 
  
%}

\end{homeworkProblem}

%----------------------------------------------------------------------------------------
%   PROBLEM 2
%----------------------------------------------------------------------------------------

\begin{homeworkProblem}

 Using the PCI book modified docclass.py code and test.py (see Slack assignment-8 channel)
Use your Training dataset to train the Naive Bayes classifier ( e.g., docclass.spamTrain() )
Use your Testing dataset to test (test.py) the Naive Bayes classifier and report the classification results.

%\problemAnswer
%{
    \begin{verbatim}\end{verbatim}
    \textbf{SOLUTION}\\
I resued the code from test.py and trained the classifier on the data set using function \emph{sample_train()}. For the purpose of classifying documents we can call function \emph{classify_document()} with file path which needs to be classified. 

On running my classifier on my testing data using function \emph{calculate_confusion_matrix()}. Table \ref{Result} shows results on testing the classifier on the data set. 
\begin{table}[]
\centering
\begin{tabular}{|l|l|l|}
\hline
\textbf{}         & \textbf{Spam} & \textbf{Not Spam} \\ \hline
\textbf{Spam}     & 3             & 7                 \\ \hline
\textbf{Not Spam} & 0             & 10                \\ \hline
\end{tabular}
\caption{Classication results on testing dataset }
\label{Result}
\end{table}
 

\begin{lstlisting}[language=python, breaklines=true]
def classify_emails():
    cl = Assignment8.docclass.naivebayes(Assignment8.docclass.getwords)
    cl.setdb('SpamClassifier.db')
    sample_train(cl)
    testing_data = "/home/msiddique/WSDL_Work/WebScience/Assignment8/Dataset/testing/"
    # classify_document(cl, testing_data + "/spam/" + "2.txt")
    calculate_confusion_matrix(cl)


def sample_train(cl):
    training_data = "/home/msiddique/WSDL_Work/WebScience/Assignment8/Dataset/training/"
    for dir, path, files in os.walk(training_data):
        for file_name in files:
            with open(dir + "/" + file_name, "r") as file_object:
                file_content = file_object.read()
                print(file_content)
                file_tag = dir.split("/")[-1]
                if file_tag == "notspam":
                    file_tag = "not spam"
                cl.train(file_content, file_tag)


def classify_document(cl, file_path):
    with open(file_path, "r")as file_object:
        file_content = file_object.read()
        return cl.classify(file_content)

\end{lstlisting}
%}

\end{homeworkProblem}

%----------------------------------------------------------------------------------------
%   PROBLEM 3
%----------------------------------------------------------------------------------------

\begin{homeworkProblem}

Draw a confusion matrix for your classification results
(see: \url{https://en.wikipedia.org/wiki/Confusion_matrix})
%\problemAnswer
%{
    \begin{verbatim}\end{verbatim}
    \textbf{SOLUTION}\\
I used the \emph{calculate_confusion_matrix()} function to classify the testing documents. Table \ref{Matrix} shows the results for confusion matrix.

\begin{table}[]
\centering
\begin{tabular}{|l|l|l|}
\hline
\textbf{}         & \textbf{Spam} & \textbf{Not Spam} \\ \hline
\textbf{Spam}     & 3   (TP)          & 7 (FP)                 \\ \hline
\textbf{Not Spam} & 0   (FN)          & 10 (TN)               \\ \hline
\end{tabular}
\caption{Results for confusion matrix on testing dataset}
\label{Matrix}
\end{table}

\begin{lstlisting}[language=python, breaklines=true]
def calculate_confusion_matrix(cl):
    fn = 0
    fp = 0
    tn = 0
    tp = 0
    testing_data = "/home/msiddique/WSDL_Work/WebScience/Assignment8/Dataset/testing/"
    for dir, path, files in os.walk(testing_data):
        for file_name in files:
            classifier_result = classify_document(cl, dir + "/" + file_name)
            file_tag = dir.split("/")[-1]
            if file_tag == "notspam":
                file_tag = "not spam"
            if file_tag == "spam" and classifier_result == "spam":
                tp += 1
            elif file_tag == "spam" and classifier_result == "not spam":
                fp += 1
            elif file_tag == "not spam" and classifier_result == "spam":
                fn += 1
            elif file_tag == "not spam" and classifier_result == "not spam":
                tn += 1

            print(classifier_result)
    print("TP: " + str(tp))
    print("FP: " + str(fp))
    print("TN: " + str(tn))
    print("FN: " + str(fn))
    print("Recall: " + str(tp/(tp + fn)))
    print("Precision: " + str(tp / (tp + fp)))
    print("Accuracy: " + str((tp + tn) / (tn + tp + fn + fp)))
\end{lstlisting}

%}

\end{homeworkProblem}

%----------------------------------------------------------------------------------------
%   PROBLEM 4
%----------------------------------------------------------------------------------------

\begin{homeworkProblem}

Report the precision and accuracy scores of your classification results
(see: \url{https://en.wikipedia.org/wiki/Precision_and_recall})
%\problemAnswer
%{
    \begin{verbatim}\end{verbatim}
    \textbf{SOLUTION}\\
$Recall =\frac{TP}{TP + FN}$ 

$Recall = \frac{3}{3 + 0}$

$Recall = 1$\\\\

$Precision = \frac{TP}{TP + FP}$

$Precision = \frac{3}{3 + 7}$

$Precision = 0.3$\\\\

$Accuracy = \frac{TP + TN}{TP + TN + FN + FP}$

$Accuracy = \frac {3 + 10}{3 + 10 + 0 + 7}$

$Accuracy = 0.65$ 

%}

\end{homeworkProblem}

\end{document}
    


   

    

    

    
   
